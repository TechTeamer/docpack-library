\documentclass[12pt, twoside]{report}
\usepackage[fencedCode,smartEllipses,hybrid]{markdown}
\usepackage[a4paper,
  inner = 25mm,
  outer = 25mm,
  top = 25mm,
  bottom = 30mm,
  footskip = 20mm
]{geometry}
\usepackage[magyar]{babel}
\usepackage{t1enc}
\usepackage[utf8]{inputenc}
\usepackage{verbatim}
\usepackage[hidelinks]{hyperref}
\usepackage{minted}
\usepackage{pmboxdraw}
\usepackage{xcolor}
\usepackage{mdframed}
\usepackage{graphicx}
\usepackage{setspace}
\usepackage{fancyhdr}
\usepackage{titlesec}
\usepackage{longtable}
\usepackage{lipsum}
\usepackage{fontspec}
\usepackage{tabularray}


\setmainfont{arial}[ %Need such long loading because of luaotfload issue
    Path = /usr/share/fonts/truetype/msttcorefonts/,
    Extension = .ttf,
    UprightFont = *,
    BoldFont = *bd,
    ItalicFont = *i,
    BoldItalicFont = *bi]
%\newfontfamily\cyrillicfont{arial}[ %for Cyrillic users
%    Extension = .ttf ,
%    UprightFont = *,
%    BoldFont = *bd,
%    ItalicFont = *i,
%    BoldItalicFont = *bi]
\onehalfspacing

\BeforeBeginEnvironment{minted}{\begin{mdframed}[backgroundcolor=bg, hidealllines=true]}
\AfterEndEnvironment{minted}{\end{mdframed}}

\definecolor{bg}{rgb}{0.95,0.95,0.95}

\titleformat{\chapter}[display]
{\normalfont\bfseries}{}{0pt}{\Huge}

\pagestyle{fancy}

\fancyhf{}
\fancyfoot[LO,RE]{\thepage}
\fancyfoot[C]{\includegraphics[width=1.7cm]{images/footer}}
\renewcommand{\headrulewidth}{0pt}
\fancypagestyle{plain}{
  \fancyhf{}
  \fancyfoot[LO,RE]{\thepage}
  \fancyfoot[C]{\includegraphics[width=1.7cm]{images/footer}}
  \renewcommand{\headrulewidth}{0pt}
}

\begin{document}
  \titleformat{\chapter}[hang]
  {\Large\bfseries}
  {}
  {0pt}
  {\LARGE\MakeUppercase}
  [\normalfont]
  \titlespacing{\chapter}{0pt}{0pt}{30pt}[0pt]

  \begin{titlepage}
    \vspace*{5cm}
    \includegraphics[width=0.9\columnwidth]{images/logo}
    \vspace*{2.5cm}
    \begin{center}
      \Huge FaceKom rendszer - \\
      Általános dokumentum
    \end{center}

    \vspace*{4.5cm}

    \begin{flushright}
      \Large Kiadás dátuma: \today

      \vspace*{2mm}

      \Large Verzió: 2.1.0

      \vspace*{2mm}

      \Large Készítette: FaceKom Kft.
    \end{flushright}
  \end{titlepage}
  \setcounter{page}{1}
  \tableofcontents

  \begin{sloppypar}
  % \input{output/usermanualhu.md}
  % \markdownInput{usermanualhu.md}
  \end{sloppypar}
  \lipsum[2-4]

    \begin{longtblr}[]{
      colspec = {|X[-1]|X|X|X|X|},
      row{1} = {c,m,gray9},
      row{even} = {c,m},
      row{odd} = {c,m},
      rowhead = 1
    }
      \hline
      \# & Esemény kategória & Esemény & Leírás & Tárolt paraméterek \\ \hline
      1 & system & exit & Rendszer leállása & posix signal \\ \hline
      2 & system & start & Rendszer elindulása & rendszer verziója \\ \hline
      3 & openhour & updated & Az általános nyitvatartás megváltozott & az új nyitvatartási rend \\ \hline
      4 & openhour & exceptioncreated & Új kivételes nyitva- vagy zárva tartás került a rendszerbe & az új nyitva- vagy zárva tartás adatai \\ \hline
      5 & openhour & exceptiondeleted & Kivételes nyitva- vagy zárva tartás törölve lett & a törölt nyitva- vagy zárva tartás adatai \\ \hline
      6 & user & login & Felhasználó bejelentkezett & a felhasználó user neve és az aktuális jogosultságai \\ \hline
      7 & user & logout & Felhasználó kijelentkezett & a felhasználó user neve \\ \hline
      8 & user & role-switch & Egy felhasználó szerepkört váltott & felhasználó szerepköre, felhasználóneve, bejelentkezési mód \\ \hline
      9 & user & auth.unknown\_user & Bejelentkezési kísérlet nem létező felhasználóval & a nem létező felhasználó neve \\ \hline
      10 & user & auth.wrong\_password & Bejelentkezési kísérlet rossz jelszóval & a felhasználó neve \\ \hline
      11 & user & auth.expired\_password & A felhasználó sikeresen belépett de a jelszava lejárt & felhasználó felhasználóneve \\ \hline
      12 & user & auth.no\_main\_role & A felhasználónak nincs megfelelő szerepköre & felhasználó felhasználóneve \\ \hline
      13 & user & auth.password\_success & Felhasználói jelszó sikeres változása & felhasználó id-ja, neve \\ \hline
      14 & user & auth.sms\_failed & SMS küldés sikertelen & felhasználó neve \\ \hline
      15 & user & auth.sms\_success & SMS küldés sikeres & felhasználó neve \\ \hline
      16 & user & auth.session\_expired & Lejárt session érzékelése & a felhasználó neve \\ \hline
      17 & user & auth.session\_destroy & Session megsemmisítése & a felhasználó neve \\ \hline
      18 & user & user.auth.throttled & A felhasználó ideiglenesen ki lett tiltva & ip cím, bejelentkezési mód \\ \hline
      19 & user & user.throttled & A felhasználó ideiglenesen ki lett tiltva & ip cím, bejelentkezési mód \\ \hline
      20 & user & auth.not\_authorized & Egy nem engedélyezett user be próbált lépni & felhasználó felhasználóneve \\ \hline
      21 & user & auth.disabled & Egy letiltott felhasználó be próbált lépni & felhasználó id-ja és neve \\ \hline
      22 & user & passwordchanged & Felhasználói jelszó megváltozott & a felhasználó id-ja és neve \\ \hline
      23 & user & passwordreseted & Felhasználó jelszavát resetelték & resetet végző felhasználó, a resetelt felhasználó \\ \hline
      24 & user & created & Új felhasználó lett létrehozva & az új felhasználó id-ja, neve, státusza, és az aktuális jogosultságai \\ \hline
      25 & user & updated & Felhasználó adatai megváltoztak & a felhasználó id-ja, neve, státusza, vezeték és kereszt neve, valamint email címe \\ \hline
      26 & user & batch.import & Felhasználói adatok batch importálása & - \\ \hline
      27 & user & changed & Felhasználó adatai megváltoztak & a felhasználó id-ja, neve, státusza, vezeték és kereszt neve, valamint email címe \\ \hline
      28 & user & changed.rights & Felhasználó jogai megváltoztak & felhasználó id-ja, neve, státusza, és az aktuális jogosultságai \\ \hline
      29 & user & enabled & Egy felhasználó aktiválva lett & felhasználó id-ja és neve \\ \hline
      30 & user & disabled & Egy felhasználó tiltva lett & felhasználó id-ja és neve \\ \hline
      31 & system & disaster.on & Vészhelyzeti rendszer leállás kezdete & - \\ \hline
      32 & system & disaster.off & Vészhelyzeti rendszer leállás vége & - \\ \hline
      33 & report & accessed & Ügyfél adat hozzáférés & oldal URL-je, ügyfél id-ja, szoba id-ja, önkiszolgáló szoba id-ja, ügymenet id-ja, visszahívási kérelem id-ja, email log id-ja, feladat id-ja \\ \hline
      34 & report & downloaded & Dokumentum letöltés & dokumentum id-ja, ügyfél id-ja, siker státusza \\ \hline
      35 & room & downloaded & A szoba le lett töltve & szoba id-ja \\ \hline
      36 & room & rawvideodownloaded & A szoba nyers videofájljai le lettek töltve & szoba id-ja \\ \hline
      37 & room & audiodownloaded & A szoba hangsávjának letöltése & szoba id-ja \\ \hline
      38 & room & deleted.record & Egy felvétel állomány törölve lett & szoba id-ja, fájl neve és a törlés sikerességének / sikertelenségének eredménye \\ \hline
      39 & room & deleted.screenshot & Egy képernyő kép állomány törölve lett & szoba id-ja, csatolmány id-ja és a törlés sikerességének / sikertelenségének eredménye \\ \hline
      40 & room & deleted & A szoba törölve lett & szoba id-ja, fájlok, törlés oka, siker státusza \\ \hline
      41 & room & archived & A szoba archiválva lett & szoba id-ja, az archiválás sikerességének / sikertelenségének eredménye, érintett fájlok \\ \hline
      42 & room & restored & A szoba vissza lett állítva (az archivált állapotból) & szoba id-ja, az visszaállítás sikerességének / sikertelenségének eredménye, érintett fájlok \\ \hline
      43 & room & reconverted & A szoba vissza lett állítva (a törölt állapotból) & szoba id-ja, a visszaállítás sikerességének / sikertelenségének eredménye, érintett fájlok \\ \hline
      44 & selfServiceRoom & reconverted & Az önkiszolgáló szoba vissza lett állítva & szoba id-ja \\ \hline
      45 & selfServiceRoom & archived & Az önkiszolgáló szoba archiválva lett & szoba id-ja \\ \hline
      46 & selfServiceRoom & restored & Az önkiszolgáló szoba vissza lett állítva (az archivált állapotból) & szoba id-ja \\ \hline
      47 & selfServiceRoom & deleted & Az önkiszolgáló szoba törölve lett & szoba id-ja \\ \hline
      48 & selfServiceRoom & downloaded & Az önkiszolgáló szoba le lett töltve & szoba id-ja \\ \hline
      49 & flow & archived & A folyamat archiválva lett & folyamat id-ja \\ \hline
      50 & flow & restored & A folyamat vissza lett állítva (az archivált állapotból) & folyamat id-ja \\ \hline
      51 & flow & draft.created & Új folyamatvázlat létrehozva & folyamat vázlat id-ja \\ \hline
      52 & flow & published & A folyamat közzétéve lett & folyamat id-ja \\ \hline
      53 & flow & cleared & A folyamat törölve lett & folyamat id-ja \\ \hline
      54 & settings & archive & Beállítások archiválása & beállítások id-ja \\ \hline
      55 & settings & viewed & Beállítások megtekintése & beállítások id-ja \\ \hline
      56 & settings & exported & Beállítások exportálása & beállítások id-ja \\ \hline
      57 & settings & imported & Beállítások importálása & beállítások id-ja \\ \hline
      58 & document & deleted & Dokumentum törölve lett & dokumentum id-ja \\ \hline
      59 & systemdocument & deleted & Rendszerdokumentum törölve lett & dokumentum id-ja \\ \hline
      60 & systemdocument & accessed & Rendszerdokumentum hozzáférve lett & dokumentum id-ja \\ \hline
      61 & systemdocument & uploaded & Rendszerdokumentum feltöltve lett & dokumentum id-ja \\ \hline
      62 & systemdocument & downloaded & Rendszerdokumentum letöltve lett & dokumentum id-ja \\ \hline
      63 & systemdocument & updated & Rendszerdokumentum frissítve lett & dokumentum id-ja \\ \hline
      64 & flow & reset & A folyamat alaphelyzetbe állítva & folyamat id-ja \\ \hline
      65 & encrypt & failed & Titkosítási művelet sikertelen & - \\ \hline
      66 & timestamp & failed & Időbélyegzés sikertelen & - \\ \hline
      67 & callcenter & changed & Call center beállítások megváltoztak & call center id-ja \\ \hline
      68 & cronjob & run & Cron feladat futtatása & cronjob id-ja \\ \hline
      69 & cronjob & start & Cron feladat elindítása & cronjob id-ja \\ \hline
      70 & cronjob & stop & Cron feladat leállítása & cronjob id-ja \\ \hline
      71 & customer & deleted & Ügyfél törölve lett & ügyfél id-ja \\ \hline
      72 & access & failed & Hozzáférés sikertelen & - \\ \hline
      73 & user & webAuthnCredential.create & WebAuthn credential létrehozva & felhasználó id-ja, credential id-ja \\ \hline
      74 & user & webAuthnCredential.delete & WebAuthn credential törölve & felhasználó id-ja, credential id-ja \\ \hline
    \end{longtblr}

%  \include{readme}
\end{document}
